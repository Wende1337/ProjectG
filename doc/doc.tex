\NeedsTeXFormat{LaTeX2e}
\documentclass[oneside]{book}%[a4paper]{book}%article%[oneside]{book}

% LaTeX-Zusatzpakete
\usepackage{german}                % deutsches Dokument (Trennung etc.)
\usepackage[utf8]{inputenc}      % deutsche Eingabezeichen (äü etc. erlaubt)
\usepackage{times}                 % Times Font
\usepackage{graphicx}              % Grafiken einbinden (hängt von latex/dvipdf oder pdflatex ab!)
\usepackage{tabularx}              % erweiterte Tabellenmöglichkeiten
\usepackage{subfig}                % Teilgrafiken erlauben
\usepackage{amsfonts}              % fuer zusätzliche math. Symbole
\usepackage{amssymb}               % weitere math. Symbole
\usepackage{theorem}               % Theorem-Umgebung
\usepackage{listings}              % Programmlistings
\usepackage{xcolor}		%for coloring
\usepackage{float}			%for placing images between text/paragraphs
% Steuerung fuer Programmlistings (anpassen an Programmiersprache)
\lstset{language=C, frame=single, breaklines=true, tabsize=2}


%###############################################################################
% Grössenangaben zum Dokument etc.

%verlinktes inhaltsverzeichnis
\usepackage[colorlinks,
pdfstartview = FitH,
bookmarksopen = true,
bookmarksnumbered = true,
linkcolor = black,
plainpages = false,
hypertexnames = false,
citecolor = black]{hyperref}


% vertikal
\setlength{\voffset}{-0.5cm}
\setlength{\textheight}{23cm}
\setlength{\topmargin}{0cm}
\setlength{\headheight}{6mm}
\setlength{\headsep}{1cm}
\setlength{\topskip}{0cm}
\setlength{\footskip}{1cm}

% horizontal
\setlength{\hoffset}{-0.4cm}
\setlength{\textwidth}{15.5cm}
\setlength{\oddsidemargin}{0.8cm}
\setlength{\evensidemargin}{0.8cm}

\setlength{\parindent}{0cm}        % kein Einzug bei Paragrafenbeginn


%###############################################################################
% Hier geht es los

% Autor und Abgabedatum ändern
\def\autor{}
\def\datum{}

% Titelseite erzeugen
\begin{document}

%###############################################################################
% Titelseite

\sloppy
\pagestyle{headings}
\bibliographystyle{alphadin}
\pagenumbering{roman}

\begin{titlepage}
 \renewcommand{\baselinestretch}{1.4}\normalsize
  
  \vspace{5cm}
  \begin{center}
    
    \begin{Huge}\textbf{Griechisch App}\end{Huge}\\
    \vspace{0.8cm}
    \begin{huge}\textbf{Learnings}\end{huge}\\
    
%    \vspace{2.2cm}
%    \renewcommand{\baselinestretch}{1.2}\normalsize
%    \begin{huge}
%      \textbf{Thema der Arbeit\\}
%    \end{huge}
%    \renewcommand{\baselinestretch}{1.5}\normalsize
%    \vspace{0.7cm}
    \vspace{0.6cm}
    
    \begin{Large}\textbf{von \autor\ \\}
    \end{Large}
  \end{center}
\end{titlepage}



\chapter{Progressive Web App}
\subsection{•}


%###############################################################################
% Inhaltsverzeichnis

\tableofcontents           % Inhaltsverzeichnis generieren
\cleardoublepage
\pagenumbering{arabic}


%###############################################################################
% Literaturverzeichnis und ggfs. Index

\cleardoublepage
\addcontentsline{toc}{chapter}{Literaturverzeichnis}
\bibliography{literatur}

% soweit gewünscht: Tabellen- und Abbildungsverzeichnis
\cleardoublepage
\listoftables              % Tabellenverzeichnis
\cleardoublepage
\listoffigures             % Abbildungsverzeichnis


%###############################################################################
% Appendix (falls nötig)

\cleardoublepage
\addcontentsline{toc}{chapter}{Anhang}
\begin{appendix}           % Anhang

\chapter{Code snippets}

\begin{enumerate}
 
 \item manifest: 
\begin{lstlisting}
{
  "name": "Weather",
  "short_name": "Weather",
  "icons": [{
    "src": "/images/icons/icon-128x128.png",
      "sizes": "128x128",
      "type": "image/png"
    }, {
      "src": "/images/icons/icon-144x144.png",
      "sizes": "144x144",
      "type": "image/png"
    }, {
      "src": "/images/icons/icon-152x152.png",
      "sizes": "152x152",
      "type": "image/png"
    }, {
      "src": "/images/icons/icon-192x192.png",
      "sizes": "192x192",
      "type": "image/png"
    }, {
      "src": "/images/icons/icon-256x256.png",
      "sizes": "256x256",
      "type": "image/png"
    }, {
      "src": "/images/icons/icon-512x512.png",
      "sizes": "512x512",
      "type": "image/png"
    }],
  "start_url": "/index.html",
  "display": "standalone",
  "background_color": "#3E4EB8",
  "theme_color": "#2F3BA2"
}

\\ verlinkung mit dem index.html
<link rel="manifest" href="/manifest.json">
\end{lstlisting}


\item ios meta tags und icons
\begin{lstlisting}
<meta name="apple-mobile-web-app-capable" content="yes">
<meta name="apple-mobile-web-app-status-bar-style" content="black">
<meta name="apple-mobile-web-app-title" content="Weather PWA">
<link rel="apple-touch-icon" href="/images/icons/icon-152x152.png">
\end{lstlisting}

\item description für SEO
\begin{lstlisting}
<meta name="description" content="A sample weather app">
\end{lstlisting}

\item theme-color
\begin{lstlisting}
<meta name="theme-color" content="#2F3BA2" />
\end{lstlisting}

\item Registrierung des Service Workers
\begin{lstlisting}
if ('serviceWorker' in navigator) {
  window.addEventListener('load', () => {
    navigator.serviceWorker.register('/service-worker.js')
        .then((reg) => {
          console.log('Service worker registered.', reg);
        });
  });
}
\end{lstlisting}


\item Array für die Sachen die es zu cachen gilt
\begin{lstlisting}
const FILES_TO_CACHE = [
  '/offline.html',
];
\end{lstlisting}

\item Precache the offline page beim install event
\begin{lstlisting}
evt.waitUntil(
    caches.open(CACHE_NAME).then((cache) => {
      console.log('[ServiceWorker] Pre-caching offline page');
      return cache.addAll(FILES_TO_CACHE);
    })
);
\end{lstlisting}

\item lösche alte cache daten beim activate event
\begin{lstlisting}
evt.waitUntil(
    caches.open(CACHE_NAME).then((cache) => {
      console.log('[ServiceWorker] Pre-caching offline page');
      return cache.addAll(FILES_TO_CACHE);
    })
);
\end{lstlisting}

\item fetch event handler
\begin{lstlisting}
if (evt.request.mode !== 'navigate') {
  // Not a page navigation, bail.
  return;
}
evt.respondWith(
    fetch(evt.request)
        .catch(() => {
          return caches.open(CACHE_NAME)
              .then((cache) => {
                return cache.match('offline.html');
              });
        })
);
\end{lstlisting}

\item prüft ob das object im cache liegt und bereit ist um geladen zu werden
\begin{lstlisting}
if (!('caches' in window)) {
  return null;
}
const url = `${window.location.origin}/forecast/${coords}`;
return caches.match(url)
    .then((response) => {
      if (response) {
        return response.json();
      }
      return null;
    })
    .catch((err) => {
      console.error('Error getting data from cache', err);
      return null;
    });
\end{lstlisting}

\item call getForecast
\begin{lstlisting}
getForecastFromCache(location.geo)
    .then((forecast) => {
      renderForecast(card, forecast);
    });
\end{lstlisting}

\item app.js
\begin{lstlisting}
// If the data on the element is newer, skip the update.
if (lastUpdated >= data.currently.time) {
  return;
}
\end{lstlisting}

\item service-worker.js Konstanten für den cache
\begin{lstlisting}
const CACHE_NAME = 'static-cache-v2';
const DATA_CACHE_NAME = 'data-cache-v1';
\end{lstlisting}

\item service-woreker.js Daten die gecached werden sollen
\begin{lstlisting}
const FILES_TO_CACHE = [
  '/',
  '/index.html',
  '/scripts/app.js',
  '/scripts/install.js',
  '/scripts/luxon-1.11.4.js',
  '/styles/inline.css',
  '/images/add.svg',
  '/images/clear-day.svg',
  '/images/clear-night.svg',
  '/images/cloudy.svg',
  '/images/fog.svg',
  '/images/hail.svg',
  '/images/install.svg',
  '/images/partly-cloudy-day.svg',
  '/images/partly-cloudy-night.svg',
  '/images/rain.svg',
  '/images/refresh.svg',
  '/images/sleet.svg',
  '/images/snow.svg',
  '/images/thunderstorm.svg',
  '/images/tornado.svg',
  '/images/wind.svg',
];
\end{lstlisting}

\item service-worker.js update fetch handler
\begin{lstlisting}
if (evt.request.url.includes('/forecast/')) {
  console.log('[Service Worker] Fetch (data)', evt.request.url);
  evt.respondWith(
      caches.open(DATA_CACHE_NAME).then((cache) => {
        return fetch(evt.request)
            .then((response) => {
              // If the response was good, clone it and store it in the cache.
              if (response.status === 200) {
                cache.put(evt.request.url, response.clone());
              }
              return response;
            }).catch((err) => {
              // Network request failed, try to get it from the cache.
              return cache.match(evt.request);
            });
      }));
  return;
}
evt.respondWith(
    caches.open(CACHE_NAME).then((cache) => {
      return cache.match(evt.request)
          .then((response) => {
            return response || fetch(evt.request);
          });
    })
);
\end{lstlisting}


\item install script verlinkung
\begin{lstlisting}
<script src="/scripts/install.js"></script>
\end{lstlisting}


\item beforeinstallprompt event
\begin{lstlisting}
window.addEventListener('beforeinstallprompt', saveBeforeInstallPromptEvent);
\end{lstlisting}

\item installation fehlt noch

\end{enumerate}



\end{appendix}


%###############################################################################

\end{document}

%###############################################################################
